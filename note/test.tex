\documentclass[a4paper,12pt]{article}
\usepackage[utf8]{inputenc}
\usepackage{amsmath, amssymb}
\usepackage{graphicx}
\usepackage[left=1.00in, right=1.00in, top=1.00in, bottom=1.00in]{geometry}
\usepackage{enumitem}
\usepackage{caption}

\title{The Social Cost of Traffic at Equilibrium: A Game-Theoretic Approach}
\author{Extracted from Chapter 8: Modeling Network Traffic Using Game Theory}
\date{}

\begin{document}

\maketitle

\section{Introduction}
This section explores how game theory models network traffic, focusing on the social cost at equilibrium versus the social optimum. A key phenomenon, the \emph{Braess Paradox}, shows that adding roads can worsen traffic, challenging intuition that network upgrades always improve outcomes.

\subsection{The Braess Paradox}
Consider a network where drivers choose paths selfishly, leading to a Nash equilibrium. Adding an edge might increase travel times for all, as seen in a simple example:
\begin{itemize}
    \item \textbf{Before}: Equilibrium travel time is 60 minutes.
    \item \textbf{After Adding Edge}: Equilibrium jumps to 80 minutes, a \( 4/3 \) increase.
\end{itemize}
Roughgarden and Tardos [18, 353] prove this is the worst-case increase with linear travel-time functions $T_e(x) = a_e x + b_e$, where $a_e$, $b_e\geqslant 0$.

\begin{figure}[h]
    \centering
    \includegraphics[width=0.6\textwidth]{braess_example} % Insert Figure 8.3 from the original document (network with travel-time functions)
    \caption{Network before and after adding an edge, illustrating Braess's Paradox (to be inserted).}
    \label{fig:braess}
\end{figure}

\subsection{Broader Context}
Traffic at equilibrium may not optimize social welfare (total travel time). We aim to:
\begin{enumerate}
    \item Prove equilibria exist in any network with linear travel times.
    \item Quantify how far equilibrium social cost deviates from the optimum.
\end{enumerate}

\section{Network Model}
A traffic network is a directed graph with:
\begin{itemize}
    \item \textbf{Nodes}: Start and destination points for drivers.
    \item \textbf{Edges}: Roads with travel-time functions \( T_e(x) = a_e x + b_e \), where \( x \) is the number of drivers.
    \item \textbf{Traffic Pattern}: Path choices for all drivers.
    \item \textbf{Social Cost}: \( \text{Social-Cost}(Z) = \sum_{\text{drivers}} \text{travel time} \), summed over all drivers in pattern \( Z \).
    \item \textbf{Social Optimum}: Pattern minimizing social cost.
    \item \textbf{Nash Equilibrium}: No driver can reduce their travel time by switching paths, given others' choices.
\end{itemize}

\begin{figure}[h]
    \centering
    \includegraphics[width=0.6\textwidth]{network_travel_times} % Insert Figure 8.3 from the original document (network with travel-time functions)
    \caption{A network with travel-time functions \( T_e(x) \) on edges (to be inserted).}
    \label{fig:network}
\end{figure}

\subsection{Example Network}
In Figure \ref{fig:network_social}, with 4 drivers from \( A \) to \( B \):
\begin{itemize}
    \item \textbf{Social Optimum}: Social cost = 28 (each driver takes 7 units).
    \item \textbf{Nash Equilibrium}: Social cost = 32 (each takes 8 units).
\end{itemize}

\begin{figure}[h]
    \centering
    \includegraphics[width=0.6\textwidth]{social_vs_equilibrium} % Insert Figure 8.4 from the original document (social optimum vs. Nash equilibrium)
    \caption{Social optimum (left) vs. Nash equilibrium (right) (to be inserted).}
    \label{fig:network_social}
\end{figure}

\section{Existence of Equilibrium}
\subsection{Best-Response Dynamics}
To find an equilibrium:
\begin{enumerate}
    \item Start with any traffic pattern.
    \item If not an equilibrium, some driver can switch to a path with less travel time.
    \item Update the pattern and repeat until no driver wants to switch.
\end{enumerate}
This process, \emph{best-response dynamics}, raises the question: does it always converge?

\subsection{Potential Energy Concept}
Define \emph{potential energy} for an edge \( e \) with \( x \) drivers:
\[
\text{Energy}(e) = T_e(1) + T_e(2) + \cdots + T_e(x)
\]
Total potential energy of a pattern \( Z \) is:
\[
\text{Energy}(Z) = \sum_{e} \text{Energy}(e)
\]
If no drivers use \( e \), \( \text{Energy}(e) = 0 \).

\subsubsection{Why Potential Energy?}
Social cost can increase or decrease with best-response steps (e.g., from 28 to 32 in the Braess example), but potential energy strictly decreases, serving as a progress measure.

\subsection{Analyzing Dynamics}
Consider a driver switching paths:
\begin{itemize}
    \item \textbf{Old Path}: Travel time = 7.
    \item \textbf{New Path}: Travel time = 5.
\end{itemize}
Potential energy change:
\begin{itemize}
    \item \textbf{Released}: \( T_e(x) \) on each edge of the old path (total = 7).
    \item \textbf{Added}: \( T_e(x+1) \) on each edge of the new path (total = 5).
    \item \textbf{Net Change}: \( 5 - 7 = -2 \) (decreases).
\end{itemize}

\begin{figure}[h]
    \centering
    \includegraphics[width=0.6\textwidth]{best_response_steps} % Insert Figure 8.5 from the original document (best-response dynamics progression)
    \caption{Steps of best-response dynamics with potential energy changes (to be inserted).}
    \label{fig:dynamics}
\end{figure}

\subsubsection{General Proof}
For any edge \( e \):
\begin{itemize}
    \item Driver leaves: \( \text{Energy}(e) \) drops by \( T_e(x) \), their old travel time.
    \item Driver joins: \( \text{Energy}(e) \) rises by \( T_e(x+1) \), their new travel time.
\end{itemize}
Net change in \( \text{Energy}(Z) \) = new time - old time. Since drivers switch only to improve (new < old), \( \text{Energy}(Z) \) decreases. With finite patterns, dynamics must stop at an equilibrium.

\section{Comparing Equilibrium to Optimum}
\subsection{Potential Energy vs. Travel Time}
For edge \( e \) with \( x \) drivers:
\begin{itemize}
    \item \textbf{Total Travel Time}: \( x T_e(x) \).
    \item \textbf{Potential Energy}: \( T_e(1) + \cdots + T_e(x) \).
\end{itemize}
Since \( T_e(x) = a_e x + b_e \):
\[
\text{Energy}(e) = a_e (1 + 2 + \cdots + x) + b_e x = \frac{a_e x (x+1)}{2} + b_e x
\]
\[
x T_e(x) = x (a_e x + b_e) = a_e x^2 + b_e x
\]
Compare:
\[
\frac{1}{2} x T_e(x) \leq \text{Energy}(e) \leq x T_e(x)
\]
Proof:
\[
\frac{a_e x (x+1)}{2} + b_e x \geq \frac{1}{2} (a_e x^2 + b_e x) \quad \text{and} \quad \leq a_e x^2 + b_e x
\]

\begin{figure}[h]
    \centering
    \includegraphics[width=0.6\textwidth]{energy_travel_comparison} % Insert Figure 8.7 from the original document (potential energy vs. total travel time)
    \caption{Potential energy (shaded) vs. total travel time (rectangle) (to be inserted).}
    \label{fig:energy}
\end{figure}

\subsection{Bounding Social Cost}
For a pattern \( Z \):
\[
\frac{1}{2} \cdot \text{Social-Cost}(Z) \leq \text{Energy}(Z) \leq \text{Social-Cost}(Z)
\]
From social optimum \( Z \) to equilibrium \( Z' \):
\begin{itemize}
    \item \( \text{Energy}(Z') \leq \text{Energy}(Z) \) (decreases in dynamics).
    \item \( \text{Social-Cost}(Z') \leq 2 \cdot \text{Energy}(Z') \leq 2 \cdot \text{Energy}(Z) \leq 2 \cdot \text{Social-Cost}(Z) \).
\end{itemize}
Thus, some equilibrium has social cost at most twice the optimum. Stronger results show \( 4/3 \) is the tight bound [18, 353].

\section{Conclusion}
\begin{itemize}
    \item Equilibria exist due to decreasing potential energy in best-response dynamics.
    \item Social cost at equilibrium is bounded (up to 2x, or 4/3x with refinement) relative to the optimum.
    \item Practical implications: Network design and tolls can mitigate inefficiencies.
\end{itemize}

\end{document}
