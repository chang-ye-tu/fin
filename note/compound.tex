\documentclass[14pt]{extarticle} 
\usepackage{amsmath,amsfonts,amssymb,amsthm,graphicx,xcolor,natbib,enumitem,booktabs,tabularx}
%\usepackage[paperwidth=126mm, paperheight=96mm, top=5mm, bottom=5mm, right=5mm, left=5mm]{geometry}
\usepackage[margin=1cm]{geometry}
\pagenumbering{gobble}

%\usepackage{unicode-math}
\usepackage[BoldFont,SlantFont]{xeCJK}  
\xeCJKsetemboldenfactor{2}
\setCJKmainfont{cwTeX Q Yuan Medium}
%\setCJKmainfont{cwTeX Q Kai Medium}
%\setCJKmainfont{cwTeX Q Ming Medium}
%\setCJKmainfont[AutoFakeSlant=.1,AutoFakeBold=1]{cwTeX Q Kai Medium} 
%\setCJKfamilyfont{kaiv}[Vertical=RotatedGlyphs]{cwTeX Q Medium}
%\setmainfont{texgyrepagella-regular.otf}
%\setmathfont{texgyrepagella-math.otf}
\newcommand{\ds}{\displaystyle}
\newcommand{\ie}{\,\Longrightarrow\,}
\newcommand{\ifff}{\,\Longleftrightarrow\,}
\newcommand{\mi}{\mathrm{i}}
\DeclareMathOperator*{\dom}{dom}
\DeclareMathOperator*{\codom}{codom}
\DeclareMathOperator*{\ran}{ran}
\newcommand{\floor}[1]{\lfloor #1 \rfloor}
\newcommand{\ceil}[1]{\lceil #1 \rceil}

% figure --> 圖
\renewcommand{\appendixname}{附錄}
\renewcommand{\figurename}{圖}
\renewcommand{\tablename}{表}
\renewcommand{\refname}{參考文獻}

\usepackage{hyperref}
\hypersetup{
    colorlinks,
    linkcolor={red!50!black},
    citecolor={blue!60!black},
    urlcolor={blue!60!black}
    %urlcolor={blue!80!black}
}

\theoremstyle{definition}
\newtheorem*{dfn}{定義}
\newtheorem*{prp}{性質}
\newtheorem*{thm}{定理}
\newtheorem*{ex}{範例}
\newtheorem*{sol}{解}
\newtheorem*{prf}{證明}

%\setenumerate{label=(\roman*),itemsep=1pt,topsep=3pt}
\newcommand{\myline}{\noindent\makebox[\linewidth]{\rule{\paperwidth}{0.4pt}}}
%\newcommand{\myline}{\textcolor[RGB]{220,220,220}{\rule{\linewidth}{1pt}}}

\usepackage{tikz}
\usetikzlibrary{arrows.meta,angles,quotes}

\renewcommand\tabularxcolumn[1]{m{#1}}

\begin{document}
\title{\texorpdfstring{\vspace{-1em} 關於連續複利}{關於連續複利}} 
\author{\vspace{-5em}}
\date{\vspace{-5em}}
\maketitle
\vspace{1cm}
%\newpage
\noindent
若起始投資金額為 $A$,投資 $n$ 年,年利率 $R$。若每年複利 $1$ 次,則投資期滿之金額為
\begin{align*}
  A\,(1 + R)^n
\end{align*}
若每年複利 $m$ 次,則投資期滿之金額為
\begin{align*}
  A\,\Big(1 + \frac{R}{m}\,\Big)^{m\cdot n}
\end{align*}
若採連續複利,亦即 $m\to\infty$,則投資期滿之金額為
\begin{align}\label{eq}
  \lim_{m\to\infty}A\,\Big(1 + \frac{R}{m}\,\Big)^{m\cdot n} = A\lim_{m\to\infty}\Big(1 + \frac{R}{m}\,\Big)^{m\cdot n}
\end{align}
由數學定義
\begin{align*}
  e = \lim_{M\to\infty}\Big(1 + \frac{1}{M}\Big)^M
\end{align*}
\eqref{eq} 可表示為
\begin{align*}
  A\lim_{m\to\infty}\Big(1 + \frac{R}{m}\,\Big)^{m\cdot n} 
  &= A\lim_{m\to\infty}\Big(1 + \frac{R}{m}\,\Big)^{\frac{m}{R}\cdot R\cdot n} \\
  &= A\lim_{\frac{m}{R}\to\infty}\Big(1 + \frac{R}{m}\,\Big)^{\frac{m}{R}\cdot R\cdot n} \\
  &= A\left(\lim_{\frac{m}{R}\to\infty}\left(1 + \frac{R}{m}\,\right)^{\frac{m}{R}}\right)^{R\cdot n}\quad(A\,x^{R\cdot n}\;\text{為 $x$ 之連續函數})\\
  &= A\left(\lim_{M\to\infty}\left(1 + \frac{1}{M}\,\right)^M\right)^{R\cdot n}\quad(\text{變數變換:} M\longleftarrow\frac{m}{R})\\
  &= A\,e^{R\cdot n} 
\end{align*}
若 $R_c$ 為連續複利情形下之利率,$R_m$ 為相對應每年複利 $m$ 次之利率,則
\begin{align*}
  e^{R_c} = \Big(1 + \frac{R_m}{m}\Big)^m
\end{align*}
上式分別兩邊取 $\ln$、兩邊開 $m$ 次方,整理可得
\begin{align*}
  R_c &= m\,\ln\Big(1 + \frac{R_m}{m}\Big) \\
  R_m &= m\,\big(e^{\frac{R_c}{m}} - 1\big)
\end{align*}

%\bibliographystyle{elsarticle-harv}
%\bibliography{e}

\end{document}
