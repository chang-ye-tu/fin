\documentclass[10pt]{beamer}

\usetheme{Madrid}
\usecolortheme{default}

% Base packages
%\usepackage{helvet}
\usepackage{amsmath,amssymb,amsthm,mathtools,subcaption}
\usepackage{tikz,pgfplots,tabularx,booktabs}
\usetikzlibrary{arrows.meta, positioning, quotes}

\usepackage{xcolor}

%\usepackage[cache=false]{minted}
%\renewcommand{\theFancyVerbLine}{\sffamily\textcolor[rgb]{0.5,0.5,1.0}{\scriptsize\oldstylenums{\arabic{FancyVerbLine}}}}
%\definecolor{bg}{rgb}{.95,.95,.95}

% Font settings
\renewcommand{\familydefault}{\sfdefault}

% TikZ libraries
\usetikzlibrary{calc,positioning,backgrounds,decorations.pathreplacing}
\pgfplotsset{compat=1.14}

% Colors
\definecolor{deepblue}{RGB}{42,39,155}
\definecolor{lightpink}{RGB}{255,240,240}
\definecolor{lightgreen}{RGB}{240,255,240}
\definecolor{lightyellow}{RGB}{255,255,240}
\definecolor{codegray}{RGB}{245,245,245}
\definecolor{codegreen}{rgb}{0,0.6,0}
\definecolor{codepurple}{rgb}{0.58,0,0.82}

% Beamer settings
\setbeamercolor{title}{fg=white,bg=deepblue}
\setbeamercolor{frametitle}{fg=white,bg=deepblue}
\setbeamercolor{section in head/foot}{fg=white,bg=deepblue}

\setbeamertemplate{footline}[text line]{%
  \parbox{\linewidth}{\vspace*{-8pt}
  %\hfill\href{https://github.com/chang-ye-tu/fin}{https://github.com/chang-ye-tu/fin}
    \hfill
   \insertframenumber~/ \inserttotalframenumber}}
\setbeamertemplate{navigation symbols}{}%[only frame symbol]

\definecolor{foo}{rgb}{.2,.2,.7}
\AtBeginSection[]{
  \begin{frame}
  \vfill
  \centering
  \begin{beamercolorbox}[sep=8pt,center,shadow=true,rounded=true]{section page}
    \usebeamerfont{title}%
    {\color{foo} \insertsectionhead}\par%
  \end{beamercolorbox}
  \vfill
  \end{frame}
}

% https://tex.stackexchange.com/questions/30423/bibliography-in-beamer
\setbeamertemplate{bibliography entry title}{}
\setbeamertemplate{bibliography entry location}{}
\setbeamertemplate{bibliography entry note}{}

\DeclareMathOperator\prb{\mathsf{P}}
\DeclareMathOperator\expc{\mathsf{E}}
\DeclareMathOperator\var{var}
\DeclareMathOperator\cov{cov}
\DeclareMathOperator\cor{corr}
\DeclareMathOperator*{\argmax}{\arg\!\max}
\DeclareMathOperator*{\argmin}{\arg\!\min}
\DeclareMathOperator\corr{corr}
\DeclareMathOperator\rk{rank}
\DeclareMathOperator\sgn{sgn}
\DeclareMathOperator{\tr}{tr}

% Blackboard bold
\renewcommand{\AA}{\mathbb A}
\newcommand{\CC}{\mathbb C}
\newcommand{\DD}{\mathbb D}
\newcommand{\EE}{\mathbb E}
\newcommand{\FF}{\mathbb F}
\newcommand{\HH}{\mathbb H}
\newcommand{\KK}{\mathbb K}
\newcommand{\NN}{\mathbb N}
\newcommand{\PP}{\mathbb P}
\newcommand{\QQ}{\mathbb Q}
\newcommand{\RR}{\mathbb R}
\newcommand{\UU}{\mathbb U}
\newcommand{\ZZ}{\mathbb Z}

\newcommand{\ie}{\;\Longrightarrow\;}
\newcommand{\ifff}{\;\Longleftrightarrow\;}
\newcommand{\ds}{\displaystyle}

\title{Introduction to Financial Models \\ Lecture 03: Surprises \& Paradoxes III}
\author{}
\date{}

\begin{document}

\begin{frame}
\titlepage
\end{frame}

%\subsection*{Outline}
%\begin{frame}
%  \tableofcontents
%\end{frame}

\begin{frame}{St. Petersburg Paradox}
  \begin{itemize}[<+->]
    \item Proposed by Nicolas Bernoulli in 1713, analyzed by Daniel Bernoulli in 1738
    \item The game:
      \begin{itemize}
        \item Flip a fair coin until it shows heads
        \item If heads appears on the first flip, win \$2
        \item If heads appears on the second flip, win \$4
        \item If heads appears on the third flip, win \$8
        \item In general, if heads appears on the $n$-th flip, win \$$2^n$
      \end{itemize}
    \item The expected value calculation:
      \begin{align*}
        \expc{X} = \sum_{n=1}^{\infty} 2^n \cdot \prb(\text{heads on flip } n) = \sum_{n=1}^{\infty} 2^n \cdot \frac{1}{2^n} = \sum_{n=1}^{\infty} 1 = \infty
      \end{align*}
    \item The paradox: The game has infinite expected value, but most people would only pay a small amount to play
    \item If people maximize expected value, they should be willing to pay any finite amount to play
  \end{itemize}
\end{frame}

\begin{frame}{Early Solutions to the Paradox}
  \begin{itemize}[<+->]
    \item Practical resolution: No casino has infinite resources
      \begin{itemize}
        \item With a capped maximum payout of $M$, $\expc{X} = \log_2(M) + \frac{M}{2^{\lfloor \log_2 M \rfloor}}$
        \item For $M = 2^{20} \approx$ \$1 million, $\expc{X} \approx$ \$21
      \end{itemize}
    \item Daniel Bernoulli's insight (1738): People value money differently
      \begin{itemize}
        \item ``The value of an item must not be based on its price, but rather on the utility it yields''
        \item Proposed logarithmic utility function: $U(w) = \log(w)$
        \item Diminishing marginal utility: Each extra dollar adds less utility
      \end{itemize}
    \item Gabriel Cramer (1728) suggested: $U(w) = \sqrt{w}$ 
    \item With logarithmic utility, expected utility is finite:
      \begin{align*}
        \expc{U(X)} = \sum_{n=1}^{\infty} \log(2^n) \cdot \frac{1}{2^n} = \sum_{n=1}^{\infty} \frac{n \log(2)}{2^n} = 2\log(2) < \infty
      \end{align*}
    \item This amounts to $\expc{U(X)} \approx$ \$1.39, explaining why people would only pay a small amount
  \end{itemize}
\end{frame}

\begin{frame}{The Expected Utility Hypothesis}
  \begin{block}{Mathematical Formulation}
    The agent prefers the r.v. $X$ to r.v. $Y$ if and only if $\expc U(X) > \expc U(Y)$, where $U:\RR\mapsto\RR$ is the agent's utility function.
  \end{block}
  \vspace{3mm}
  \onslide<+-> 
  \begin{itemize}[<+->]
    \item Expected utility theory attempts to explain how people make decisions under uncertainty
    \item It seeks to address paradoxes like St. Petersburg by incorporating risk preferences
    \item Key insight: people care about the utility of outcomes, not just monetary values
    \item Decision-making is based on:
      \begin{enumerate}
        \item Probabilities of different outcomes
        \item Subjective valuation (utility) of those outcomes
      \end{enumerate}
  \end{itemize}
\end{frame}

\begin{frame}{Properties of Utility Functions}
  \begin{itemize}[<+->]
    \item Three common assumptions about utility functions:
      \begin{itemize}
        \item \textbf{More is better}: $U'(w) > 0$ (monotonically increasing)
        \item \textbf{Risk aversion}: $U''(w) < 0$ (concave function)
        \item \textbf{Decreasing absolute risk aversion}: $-\frac{U''(w)}{U'(w)}$ decreases as $w$ increases
      \end{itemize}
    \item Commonly used utility functions:
      \begin{itemize}
        \item Logarithmic: $U(w) = \log(w)$
        \item Power utility: $U(w) = \frac{w^{1-\gamma} - 1}{1-\gamma}$ for $\gamma > 0, \gamma \neq 1$
        \item Exponential utility: $U(w) = -e^{-\gamma w}$ for $\gamma > 0$
        \item Quadratic utility: $U(w) = w - \frac{\gamma}{2}w^2$ for $w < \frac{1}{\gamma}$
      \end{itemize}
    \item The parameter $\gamma$ reflects the degree of risk aversion
    \item For power utility, $\gamma = 1$ corresponds to logarithmic utility (by L'Hôpital's rule)
  \end{itemize}
\end{frame}

\begin{frame}{Risk Aversion and Risk Premium I}
  \begin{itemize}[<+->]
    \item Risk Aversion
      \begin{itemize}
        \item The \textbf{risk aversion} is the preference for a certain amount over a gamble with the same expected value
        \item Example: Preferring \$50 with certainty over a 50\% chance of \$100 (and 50\% chance of \$0)
        \item Mathematically represented by a concave utility function: $U''(w) < 0$
      \end{itemize}
    \item Risk Premium 
      \begin{itemize}
        \item Let $\widetilde{X}$ be a random variable with $\expc{\widetilde{X}} = 0$ (a fair gamble)
        \item A risk-averse person would pay to avoid this gamble
        \item The \textbf{risk premium} $\pi$ is the maximum amount they would pay:
          \begin{align}\label{eq:risk_premium}
            U(w - \pi) = \expc{U(w + \widetilde{X})}
          \end{align}
      \end{itemize}
  \end{itemize}
\end{frame}

\begin{frame}{Risk Aversion and Risk Premium II}
  \begin{itemize}[<+->]
    \item For small risks, we can use Taylor expansion 
      \begin{align*}
        U(w + \widetilde{X}) &\approx U(w) + U'(w)\widetilde{X} + \frac{1}{2}U''(w)\widetilde{X}^2
      \end{align*}
    \item By $\expc{\widetilde{X}} = 0$ and $\var{\widetilde{X}} = \expc{\widetilde{X}^2} - (\expc{\widetilde{X}})^2 = \expc{\widetilde{X}^2}$, 
      \begin{align*}
        \expc{U(w + \widetilde{X})} &\approx U(w) + U'(w)\expc{\widetilde{X}} + \frac{1}{2}U''(w)\expc{\widetilde{X}^2} = U(w) + \frac{1}{2}U''(w)\var{\widetilde{X}}
      \end{align*}
    \item Similarly, for small premium, one can use Taylor expansion 
      \begin{align*}
        U(w - \pi) \approx U(w) - \pi U'(w)
      \end{align*}
    \item Substitute into the risk premium formula \eqref{eq:risk_premium} $U(w - \pi) = \expc{U(w + \widetilde{X})}$,
      \begin{align*}
        U(w) - \pi U'(w) = U(w) + \frac{1}{2}U''(w)\var{\widetilde{X}} \ie \pi = \frac{1}{2}\left(-\frac{U''(w)}{U'(w)}\right)\var{\widetilde{X}}
      \end{align*}
  \end{itemize}
\end{frame}

\begin{frame}{Risk Aversion and Risk Premium - III}
    \begin{itemize}[<+->]
      \item The term $-\frac{U''(w)}{U'(w)}$ is the coefficient of absolute risk aversion (ARA)
        \begin{itemize}
          \item Measures risk aversion in absolute dollar terms
          \item Logarithmic utility: ARA = $\frac{1}{w}$ (decreasing with wealth)
          \item Power utility: ARA = $\frac{\gamma}{w}$ (decreasing with wealth)
          \item Exponential utility: ARA = $\gamma$ (constant regardless of wealth)
        \end{itemize}
      \item The term $-\frac{U''(w)}{U'(w)} \cdot w$ is the coefficient of relative risk aversion (RRA)
        \begin{itemize}
          \item Measures risk aversion relative to wealth level
          \item Logarithmic utility: RRA = 1 (constant)
          \item Power utility: RRA = $\gamma$ (constant)
          \item Exponential utility: RRA = $\gamma w$ (increasing with wealth)
        \end{itemize}
    \end{itemize}
\end{frame}

\begin{frame}{Axiomatic Foundation of Expected Utility}
  \begin{itemize}[<+->]
    \item John von Neumann and Oskar Morgenstern (1947) provided axioms for expected utility
    \item Four key axioms:
      \begin{itemize}
        \item \textbf{Completeness}: For any lotteries $L_1$ and $L_2$, either $L_1 \succeq L_2$ or $L_2 \succeq L_1$ or both
        \item \textbf{Transitivity}: If $L_1 \succeq L_2$ and $L_2 \succeq L_3$, then $L_1 \succeq L_3$
        \item \textbf{Continuity}: If $L_1 \succeq L_2 \succeq L_3$, then there exists a probability $p \in [0, 1]$ such that $L_2 \sim p L_1 + (1-p) L_3$
        \item \textbf{Independence}: For any lotteries $L_1$, $L_2$, $L_3$ and any probability $p \in (0, 1]$, $L_1 \succeq L_2$ if and only if $p L_1 + (1-p) L_3 \succeq p L_2 + (1-p) L_3$
      \end{itemize}
    \item These axioms lead to the expected utility representation:
      \begin{align*}
        L_1 \succeq L_2 \iff \expc_{L_1}[U(x)] \geq \expc_{L_2}[U(x)]
      \end{align*}
    \item The independence axiom is particularly important and controversial
      \begin{itemize}
        \item It states that preferences between lotteries should not be affected by mixing them with a third lottery
        \item This axiom is violated in several famous paradoxes
      \end{itemize}
  \end{itemize}
\end{frame}

\begin{frame}{Allais Paradox}
  \begin{itemize}[<+->]\setlength\itemsep{0em}
    \item Game A
      \begin{align*}
        X = \begin{cases}101 & \text{prob. } 0.33\\100 & \text{prob. } 0.66\\ 0 & \text{prob. } 0.01 \end{cases} \qquad
        Y = 100 \;\text{ with prob. } 1
      \end{align*}
      \onslide<+->
      Mostly prefer $Y$ to $X$: from the Expected Utility Hypothesis 
      \onslide<+->
      \begin{multline}
        U(100) > 0.33\cdot U(101) + 0.66\cdot U(100) + 0.01\cdot U(0)\\ \ie 0.34\cdot U(100) > 0.33\cdot U(101) + 0.01\cdot U(0)
      \end{multline}
    \item Game B 
      \begin{align*}
        X = \begin{cases}100 & \text{prob. } 0.34\\ 0 & \text{prob. } 0.66 \end{cases} \qquad
        Y = \begin{cases}101 & \text{prob. } 0.33\\ 0 & \text{prob. } 0.67 \end{cases}
      \end{align*}
      \onslide<+->
      Mostly prefer $Y$ to $X$: from the Expected Utility Hypothesis 
      \onslide<+->
      \begin{multline}
        0.33\cdot U(101) + 0.67\cdot U(0) > 0.34\cdot U(100) + 0.66\cdot U(0)\\ \ie 0.33\cdot U(101) + 0.01\cdot U(0) > 0.34\cdot U(100)
      \end{multline}
  \end{itemize}
\end{frame}

\begin{frame}{Ellsberg Paradox}
  \begin{itemize}[<+->]
    \item Given an urn with $30$ balls of colors red, yellow, and black
    \item There are $10$ red balls; total $20$ yellow / black balls, but the number of each type unknown
    \item The agent estimates the probability of drawing yellow as $p$ where $0 < p < \frac{2}{3}$
    \item A single ball is drawn from the urn
  \end{itemize}
\end{frame}

\begin{frame}{Ellsberg Paradox (Cont'd)}
  \begin{itemize}[<+->]\setlength\itemsep{0em}
    \item Game A  
      \begin{align*}
        X = \begin{cases}100 & \text{if red}\\0 & \text{if yellow or black}\end{cases} \qquad
        Y = \begin{cases}100 & \text{if yellow}\\0 & \text{if red or black}\end{cases} \qquad
      \end{align*}
      \onslide<+->
      Mostly prefer $X$ to $Y$: from the Expected Utility Hypothesis 
      \onslide<+->
      \begin{multline}
        \frac{1}{3}\cdot U(100) + \frac{2}{3}\cdot U(0) > p\cdot U(100) + (1 - p)\cdot U(0) \\ \ie\big(\frac{1}{3} - p\big)\cdot U(100) > \big(\frac{1}{3} - p\big)\cdot U(0)
      \end{multline}
    \item Game B 
      \begin{align*}
        X = \begin{cases}100 & \text{if red or black}\\ 0 & \text{if yellow}\end{cases} \qquad
        Y = \begin{cases}100 & \text{if yellow or black}\\ 0 & \text{if red}\end{cases}
      \end{align*}
      \onslide<+->
      Mostly prefer $Y$ to $X$: from the Expected Utility Hypothesis 
      \onslide<+->
      \begin{multline}
        \frac{2}{3}\cdot U(100) + \frac{1}{3}\cdot U(0) > (1 - p)\cdot U(100) + p\cdot U(0) \\ \ie\big(\frac{1}{3} - p\big)\cdot U(0) > \big(\frac{1}{3} - p\big)\cdot U(100)
      \end{multline}
  \end{itemize}
\end{frame}

\end{document}
