\documentclass[10pt]{beamer}

\usetheme{Madrid}
\usecolortheme{default}

% Base packages
%\usepackage{helvet}
\usepackage{amsmath,amssymb,amsthm,mathtools,subcaption}
\usepackage{tikz,pgfplots,tabularx,booktabs}
\usetikzlibrary{arrows.meta, positioning, quotes}

\usepackage{xcolor}

%\usepackage[cache=false]{minted}
%\renewcommand{\theFancyVerbLine}{\sffamily\textcolor[rgb]{0.5,0.5,1.0}{\scriptsize\oldstylenums{\arabic{FancyVerbLine}}}}
%\definecolor{bg}{rgb}{.95,.95,.95}

% Font settings
\renewcommand{\familydefault}{\sfdefault}

% TikZ libraries
\usetikzlibrary{calc,positioning,backgrounds,decorations.pathreplacing}
\pgfplotsset{compat=1.14}

% Colors
\definecolor{deepblue}{RGB}{42,39,155}
\definecolor{lightpink}{RGB}{255,240,240}
\definecolor{lightgreen}{RGB}{240,255,240}
\definecolor{lightyellow}{RGB}{255,255,240}
\definecolor{codegray}{RGB}{245,245,245}
\definecolor{codegreen}{rgb}{0,0.6,0}
\definecolor{codepurple}{rgb}{0.58,0,0.82}

% Beamer settings
\setbeamercolor{title}{fg=white,bg=deepblue}
\setbeamercolor{frametitle}{fg=white,bg=deepblue}
\setbeamercolor{section in head/foot}{fg=white,bg=deepblue}

\setbeamertemplate{footline}[text line]{%
  \parbox{\linewidth}{\vspace*{-8pt}
  %\hfill\href{https://github.com/chang-ye-tu/fin}{https://github.com/chang-ye-tu/fin}
    \hfill
   \insertframenumber~/ \inserttotalframenumber}}
\setbeamertemplate{navigation symbols}{}%[only frame symbol]

\definecolor{foo}{rgb}{.2,.2,.7}
\AtBeginSection[]{
  \begin{frame}
  \vfill
  \centering
  \begin{beamercolorbox}[sep=8pt,center,shadow=true,rounded=true]{section page}
    \usebeamerfont{title}%
    {\color{foo} \insertsectionhead}\par%
  \end{beamercolorbox}
  \vfill
  \end{frame}
}

% https://tex.stackexchange.com/questions/30423/bibliography-in-beamer
\setbeamertemplate{bibliography entry title}{}
\setbeamertemplate{bibliography entry location}{}
\setbeamertemplate{bibliography entry note}{}

\DeclareMathOperator\prb{\mathsf{P}}
\DeclareMathOperator\expc{\mathsf{E}}
\DeclareMathOperator\var{var}
\DeclareMathOperator\cov{cov}
\DeclareMathOperator\cor{corr}
\DeclareMathOperator*{\argmax}{\arg\!\max}
\DeclareMathOperator*{\argmin}{\arg\!\min}
\DeclareMathOperator\corr{corr}
\DeclareMathOperator\rk{rank}
\DeclareMathOperator\sgn{sgn}
\DeclareMathOperator{\tr}{tr}

% Blackboard bold
\renewcommand{\AA}{\mathbb A}
\newcommand{\CC}{\mathbb C}
\newcommand{\DD}{\mathbb D}
\newcommand{\EE}{\mathbb E}
\newcommand{\FF}{\mathbb F}
\newcommand{\HH}{\mathbb H}
\newcommand{\KK}{\mathbb K}
\newcommand{\NN}{\mathbb N}
\newcommand{\PP}{\mathbb P}
\newcommand{\QQ}{\mathbb Q}
\newcommand{\RR}{\mathbb R}
\newcommand{\UU}{\mathbb U}
\newcommand{\ZZ}{\mathbb Z}

\newcommand{\ie}{\;\Longrightarrow\;}
\newcommand{\ifff}{\;\Longleftrightarrow\;}
\newcommand{\ds}{\displaystyle}

\title{Basic Problems in Quantitative Finance}
\author{}
\date{}

\begin{document}

\begin{frame}
\titlepage
\end{frame}

%\subsection*{Outline}
%\begin{frame}
%  \tableofcontents
%\end{frame}
\begin{frame}
\titlepage
\end{frame}

\begin{frame}{Basic Problems in Quantitative Finance}
  \begin{itemize}[<+->]
    \item Three core problems in quantitative finance:
      \begin{itemize}
        \item \textbf{Portfolio optimization}: Allocating wealth among risky financial assets with uncertain returns to achieve an optimal risk-reward tradeoff while satisfying various constraints (budget, exposure limits, etc.)
        \item \textbf{Risk measurement and management}: Quantifying exposure to different risk factors, developing appropriate metrics, and implementing strategies to mitigate or control risk levels within financial and non-financial firms
        \item \textbf{Asset pricing}: Determining fair values of financial instruments, especially derivatives, to detect arbitrage opportunities and support trading decisions based on theoretical models rather than simply market consensus
      \end{itemize}
    \item These problems are interconnected: pricing models inform portfolio decisions, risk measures help in both optimization and pricing, while optimal portfolios determine market equilibrium prices
  \end{itemize}
\end{frame}

\begin{frame}{Portfolio Optimization}
  \begin{itemize}[<+->]
    \item Optimization model requires three essential components:
      \begin{itemize}
        \item Decision variables $\mathbf{x} \in \mathbb{R}^n$ (e.g., portfolio weights, number of shares, or monetary amounts)
        \item Constraints defining feasible set $S \subseteq \mathbb{R}^n$ (budget restrictions, exposure limits, regulatory requirements)
        \item Objective function $f(\cdot)$ to minimize or maximize (balancing return and risk)
      \end{itemize}
    \item Key dimensions affecting portfolio problem formulation:
      \begin{itemize}
        \item Role of time: Static (single decision point) vs. dynamic (sequence of decisions adjusting to market conditions)
        \item Hierarchical level: Strategic (asset class allocation) vs. tactical (security selection) vs. operational (trade execution)
        \item Asset-only vs. asset-liability management (e.g., pension fund matching future obligations)
        \item Market model complexity: Transaction costs, liquidity constraints, price impact of trades
        \item Uncertainty representation: Probabilistic, scenario-based, robust, or ambiguity-aware approaches
        \item Risk-reward tradeoff approach: Utility functions, risk measures (VaR, CVaR), multi-objective formulations
      \end{itemize}
  \end{itemize}
\end{frame}

\begin{frame}{Static Portfolio Optimization: Mean-Variance Efficiency}
  \begin{itemize}[<+->]
    \item Portfolio weights: $w_i, i = 1, \ldots, n$, represent fraction of wealth allocated to each asset, collected in vector $\mathbf{w} \in \mathbb{R}^n$
    \item Basic constraints in the Markowitz framework:
      \begin{align*}
        \sum_{i=1}^{n} w_i \equiv \mathbf{1}^T\mathbf{w} &= 1 \quad \text{(full investment constraint)}\\
        w_i &\geqslant 0, \quad i = 1, \ldots, n \quad \text{(no short-selling constraint)}
      \end{align*}
    \item Additional practical constraints often added:
      \begin{align*}
        0.05 \leqslant \sum_{i \in I} w_i \leqslant 0.25 \quad \text{(sector/geographic exposure limits)}
      \end{align*}
    \item Portfolio return is random: $R_p = \sum_{i=1}^{n} w_iR_i = \mathbf{w}^T\mathbf{R}$
      \begin{itemize}
        \item Expected return: $\mu_p = \expc{R_p} = \sum_{i=1}^{n} w_i\mu_i = \mathbf{w}^T\boldsymbol{\mu}$ where $\mu_i = \expc{R_i}$
        \item Variance: $\sigma_p^2 = \var{R_p} = \sum_{i=1}^{n}\sum_{j=1}^{n} w_i\sigma_{ij}w_j = \mathbf{w}^T\boldsymbol{\Sigma}\mathbf{w}$
        \item Covariance: $\sigma_{ij} = \cov(R_i, R_j)$ is crucial for diversification benefits
      \end{itemize}
  \end{itemize}
\end{frame}

\begin{frame}{Mean-Variance Approaches}
  \begin{itemize}[<+->]
    \item Maximizing expected return alone leads to trivial, undiversified solution:
      \begin{align*}
        \max \quad & \boldsymbol{\mu}^T\mathbf{w}\\
        \text{s.t.} \quad & \mathbf{1}^T\mathbf{w} = 1\\
        & \mathbf{w} \geqslant \mathbf{0}
      \end{align*}
      This would allocate 100\% to the asset with highest expected return
    \item Three approaches to balance risk-reward tradeoff:
      \begin{enumerate}
        \item Linear combination (risk-adjusted return): 
        \begin{align*}
            \max \boldsymbol{\mu}^T\mathbf{w} - \lambda \cdot \mathbf{w}^T\mathbf{\Sigma}\mathbf{w}
        \end{align*}
        where $\lambda > 0$ represents risk aversion coefficient
        
        \item Risk-constrained (maximize return under risk budget): 
        \begin{align*}
            \max \boldsymbol{\mu}^T\mathbf{w} \quad \text{s.t.} \quad \mathbf{w}^T\mathbf{\Sigma}\mathbf{w} \leqslant \beta
        \end{align*}
        
        \item Return-constrained (minimize risk with return target): 
        \begin{align*}
            \min \mathbf{w}^T\mathbf{\Sigma}\mathbf{w} \quad \text{s.t.} \quad \boldsymbol{\mu}^T\mathbf{w} \geqslant \mu_{min}
        \end{align*}
      \end{enumerate}
    \item The last approach generates the efficient frontier by varying $\mu_{min}$
  \end{itemize}
\end{frame}

\begin{frame}{Dynamic Decision-Making}
  \begin{itemize}[<+->]
    \item Elements of a dynamic portfolio model:
      \begin{itemize}
        \item Control/decision variables $\mathbf{x}_t$, $t = 0, \ldots, T-1$ (portfolio weights or trading decisions)
        \item State variables $\mathbf{S}_t$, $t = 0, \ldots, T$ (wealth level, holdings, market conditions)
        \item Exogenous risk factors $\boldsymbol{\xi}_t(\omega)$, $t = 1, \ldots, T$ (market returns, interest rates)
        \item Cost/reward functions $f_t$ and terminal function $F_T$ (transaction costs, utility of wealth)
      \end{itemize}
    \item Timing convention: Observe state $\mathbf{S}_t$ → Make decision $\mathbf{x}_t$ → Random factors $\boldsymbol{\xi}_{t+1}$ realized → New state $\mathbf{S}_{t+1}$
    \item Generic stochastic optimization problem:
      \begin{align*}
        \min \quad & E\left[\sum_{t=0}^{T-1} f_t(\mathbf{S}_t, \mathbf{x}_t) + F_T(\mathbf{S}_T)\right]\\
        \text{s.t.} \quad & \mathbf{x}_t \in A_t(\mathbf{S}_t) \quad \text{(feasible actions)}\\
        & \mathbf{S}_{t+1} = \Phi_t(\mathbf{S}_t, \mathbf{x}_t, \boldsymbol{\xi}_{t+1}(\omega)) \quad \text{(state transitions)}
      \end{align*}
    \item Consumption-saving problems may include utility functions with discount factors: $\max E\left[\sum_{t=0}^{T} \beta^t u(C_t)\right]$
  \end{itemize}
\end{frame}

\begin{frame}{Risk Measurement and Management}
  \begin{itemize}[<+->]
    \item Risk measurement challenges and approaches:
      \begin{itemize}
        \item Defining a suitable risk measure:
          \begin{itemize}
            \item Symmetric (standard deviation) vs. asymmetric (VaR, CVaR)
            \item Coherence properties (subadditivity, monotonicity, etc.)
            \item Axiomatic foundations vs. practical interpretability
          \end{itemize}
        \item Creating a risk model:
          \begin{itemize}
            \item Factor models to reduce estimation burden
            \item Time-varying volatility and correlations
            \item Fat-tailed and skewed distributions
          \end{itemize}
        \item Managing risk through:
          \begin{itemize}
            \item Hedging (perfect, minimum-variance, immunization)
            \item Diversification across assets, sectors, regions
            \item Insurance and derivatives for tail risk protection
          \end{itemize}
      \end{itemize}
    \item Risk models can be classified as:
      \begin{itemize}
        \item Linear: $R_i = \alpha_i + \sum_{k=1}^{m} \beta_{ik}F_k + \epsilon_i$ (factor models)
        \item Nonlinear: Option pricing, interest rate models, default models
      \end{itemize}
  \end{itemize}
\end{frame}

\begin{frame}{Linear Risk Factor Models}
  \begin{itemize}[<+->]
    \item Simple linear factor model for asset returns:
      \vspace{-2mm}
      \begin{align*}
        R_i = \alpha_i + \sum_{k=1}^{m} \beta_{ik}F_k + \epsilon_i, \quad i = 1, \ldots, n
      \end{align*}
      \vspace{-4mm}
      \begin{itemize}
        \item $\alpha_i$ is the asset-specific intercept
        \item $\beta_{ik}$ is the sensitivity of asset $i$ to factor $k$
        \item $F_k$ is the common risk factor (market, size, value, momentum, etc.)
        \item $\epsilon_i$ is the idiosyncratic (asset-specific) risk
      \end{itemize}
      
    \item Induced covariance structure:
      \begin{align*}
        \sigma_{ij} &= \sum_{k=1}^{m} \beta_{ik}\beta_{jk}\sigma_k^2, \quad i \neq j \quad \text{(cross-asset covariance)}\\
        \sigma_i^2 &= \sum_{k=1}^{m} \beta_{ik}^2\sigma_k^2 + \sigma_{\epsilon_i}^2 \quad \text{(asset variance)}
      \end{align*}
      \vspace{-3mm}
    \item Benefits of factor models:
      \begin{itemize}
        \item Reduces parameter estimation burden: $m \times n + m + n$ vs. $\frac{n(n+1)}{2}$
        \item Provides insights into risk structure and factor exposures
        \item Separates common (systematic) and specific (diversifiable) risk
        \item Allows for targeted risk factor hedging and exposure management
      \end{itemize}
  \end{itemize}
\end{frame}

\begin{frame}{Nonlinear Risk Factor Models}
  \begin{itemize}[<+->]
    \item Nonlinear models needed for:
      \begin{itemize}
        \item Derivatives pricing (options, swaps, structured products)
        \item Bond pricing and interest rate risk (duration, convexity)
        \item Credit risk (default probabilities, recovery rates)
        \item Volatility risk (option vega exposure)
      \end{itemize}
    \item Zero-coupon bond example illustrates nonlinear interest rate risk:
      \begin{align*}
        P_z(0; R_{0,T}, T) = \tfrac{F}{(1 + R_{0,T})^T} \quad \text{(price as function of yield)}
      \end{align*}
      where $F$ is face value, $R_{0,T}$ is yield-to-maturity, and $T$ is time to maturity
      
    \item Interest rate risk increases nonlinearly with maturity:
      \begin{itemize}
        \item 3-year zero: -2.83\% loss on 1\% rate increase (from 4\% to 5\%)
        \item 10-year zero: -9.13\% loss on 1\% rate increase
        \item 30-year zero: -24.96\% loss on 1\% rate increase
      \end{itemize}
    
    \item For nonlinear exposures, first and second-order sensitivities are used:
      \begin{itemize}
        \item Duration and convexity for bonds
        \item Delta, gamma, vega, etc. for options
      \end{itemize}
  \end{itemize}
\end{frame}

\begin{frame}{Value-at-Risk (VaR)}
  \begin{itemize}[<+->]
    \item Standard deviation limitations as a risk measure:
      \begin{itemize}
        \item Symmetric risk measure (equally penalizes upside and downside)
        \item Assumes normal distributions (underestimates tail risk)
        \item Not expressed in monetary terms (harder for management to interpret)
        \item Doesn't account for skewness and fat tails in financial returns
      \end{itemize}
    \item Value-at-Risk (VaR) definition:
      \begin{align*}
        \prb\{L_H \leqslant \text{VaR}_{1-\alpha,H}\} = 1 - \alpha
      \end{align*}
      where $L_H$ is loss over horizon $H$, $\alpha$ is significance level (e.g., 5\% or 1\%)
      
    \item Loss definitions:
      \begin{itemize}
        \item Absolute loss: $L_H^a = W_0 - W_H = -W_0R_H$ (relative to initial wealth)
        \item Relative loss: $L_H^r = \expc{W_H} - W_H = W_0(\mu - R_H)$ (relative to expected wealth)
        \item For short horizons, drift is negligible and both definitions converge
      \end{itemize}
    
    \item VaR estimation approaches:
      \begin{itemize}
        \item Parametric (analytical): Based on distributional assumptions
        \item Historical simulation: Based on empirical distribution
        \item Monte Carlo: Based on simulated scenarios
      \end{itemize}
  \end{itemize}
\end{frame}

\begin{frame}{Risk Management - Hedging}
  \begin{itemize}[<+->]
    \item Perfect hedging (complete risk elimination):
      \begin{itemize}
        \item Forward contract example with $N$ units of asset:
          \begin{align*}
            N \cdot S_T + N \cdot (F_0 - S_T) = N \cdot F_0
          \end{align*}
        \item Completely eliminates risk but also eliminates all upside potential
        \item Often impractical due to maturity mismatches, basis risk, liquidity issues
      \end{itemize}
      
    \item Minimum variance hedging (cross-hedging):
      \begin{align*}
        h^* = \tfrac{\text{Cov}(S_{T_1}, F_{T_1})}{\var(F_{T_1})} = \rho \cdot \tfrac{\sigma_S}{\sigma_F}
      \end{align*}
      where $h^*$ is optimal hedge ratio, $\rho$ is correlation, $\sigma_S$ and $\sigma_F$ are volatilities
      
    \item First-order immunization (sensitivity-based):
      \begin{align*}
        \tfrac{\partial V}{\partial R_i} + \sum_{j=1}^m \phi_j \tfrac{\partial H_j}{\partial R_i} = 0, \quad i = 1, \ldots, m
      \end{align*}
      where $V$ is portfolio value, $H_j$ is hedging instrument value, $\phi_j$ is position size
      
    \item Challenges in hedging implementation:
      \begin{itemize}
        \item Standardized contract sizes causing over/under hedging
        \item Time-varying correlations and volatilities
        \item Marking-to-market impact on futures hedges
        \item Transaction costs and liquidity constraints
      \end{itemize}
  \end{itemize}
\end{frame}

\begin{frame}{Financial vs. Nonfinancial Risk Factors}
  \begin{itemize}[<+->]
    \item Financial risk factors (directly related to market variables):
      \begin{itemize}
        \item Market risk (equity prices): Systematic and idiosyncratic components
        \item Interest rate risk: Level, slope, and curvature of yield curve
        \item Currency risk: Exchange rate fluctuations affecting international operations
        \item Inflation risk: Eroding purchasing power and real returns
        \item Volatility risk: Impacting derivative pricing and hedging effectiveness
        \item Credit/counterparty risk: Default probability and recovery rates
      \end{itemize}
      
    \item Nonfinancial risk factors (often harder to quantify):
      \begin{itemize}
        \item Model risk: Wrong model specification or parameter estimation
        \item Regulatory and political risk: Changes in laws and regulations
        \item Volume risk: Uncertain business demand affecting hedging needs
        \item Operational risk: System failures, human errors, process breakdowns
        \item Liquidity risk: Unable to execute trades at fair prices
        \item Execution uncertainty: Delays between decision and implementation
      \end{itemize}
      
    \item Complex risk exposures often involve interaction between multiple risk factors
    \item Need for integrated risk management across different risk categories
  \end{itemize}
\end{frame}

\begin{frame}{No-Arbitrage Principle in Asset Pricing}
  \begin{itemize}[<+->]
    \item Arbitrage opportunity: Trading strategy that:
      \begin{itemize}
        \item Requires no initial investment ($V(0) = 0$)
        \item Guarantees non-negative payoff in all states ($V(T,\omega) \geqslant 0, \forall \omega$)
        \item Has positive expected payoff ($\expc{V(T,\omega)} > 0$)
      \end{itemize}
      
    \item No-arbitrage principle: In well-functioning markets, arbitrage opportunities cannot persist
    
    \item Uses of asset pricing models based on no-arbitrage:
      \begin{itemize}
        \item Calibrate models to observed prices to infer unobservable parameters (e.g., implied volatility)
        \item Detect relative price discrepancies for trading strategies
        \item Price OTC derivatives consistently with exchange-traded instruments
        \item Model risk factor sensitivities for hedging and risk management
        \item Decompose complex securities into simpler building blocks
      \end{itemize}
      
    \item Advantages over equilibrium models:
      \begin{itemize}
        \item Does not require assumptions about investor preferences
        \item Simpler mathematical structure
        \item Direct practical applications in trading and hedging
      \end{itemize}
  \end{itemize}
\end{frame}

\begin{frame}{Types of Arbitrage}
  \begin{itemize}[<+->]
    \item Instantaneous arbitrage (spatial arbitrage):
      \begin{itemize}
        \item Trading at a single time instant across different markets
        \item Example: Stock priced at €50 on one exchange and \$68 on another when exchange rate is \$1.34/€
        \item Triangular currency arbitrage: Exploiting inconsistencies among three exchange rates
        \item Typically eliminated very quickly through high-frequency trading
      \end{itemize}
      
    \item Static arbitrage (intertemporal):
      \begin{itemize}
        \item Trading at two time instants ($t=0$ and $t=T$)
        \item Example: Put-call parity violations
          \begin{align*}
            C_0 - P_0 = S_0 - K \cdot e^{-rT}
          \end{align*}
        \item Cash-and-carry arbitrage: Exploiting misalignments between spot and forward prices
        \item Law of one price: Identical cash flows must have identical prices
      \end{itemize}
      
    \item Dynamic arbitrage:
      \begin{itemize}
        \item Trading at multiple time instants, adjusting positions as uncertainty unfolds
        \item Example: Delta-hedging strategies in option replication
        \item More complex to implement due to transaction costs, model risk, execution uncertainty
      \end{itemize}
  \end{itemize}
\end{frame}

\begin{frame}{Option Pricing in Binomial Model}
  \begin{itemize}[<+->]
    \item Single-step binomial model:
      \begin{itemize}
        \item Current price $S_0$ at $t=0$
        \item Future prices at $t=T$: $uS_0$ (up state) or $dS_0$ (down state) with $d < u$
        \item Probability of up state $p_u$ and down state $p_d = 1-p_u$
        \item Risk-free rate $r$ (continuous compounding) with $B(T) = e^{rT}$
      \end{itemize}
      
    \item Option payoffs at maturity:
      \begin{itemize}
        \item Call: $f_u = \max\{0, uS_0-K\}$ and $f_d = \max\{0, dS_0-K\}$
        \item Put: $f_u = \max\{0, K-uS_0\}$ and $f_d = \max\{0, K-dS_0\}$
      \end{itemize}
      
    \item Replicating portfolio with $\Delta$ shares and $\Psi$ cash:
      \begin{align*}
        \Delta S_0u + \Psi e^{rT} &= f_u\\
        \Delta S_0d + \Psi e^{rT} &= f_d
      \end{align*}
      
    \item Solution for replicating portfolio:
      \begin{align*}
        \Delta &= \tfrac{f_u - f_d}{S_0(u - d)}\\
        \Psi &= e^{-rT} \cdot \tfrac{uf_d - df_u}{u - d}
      \end{align*}
      
    \item Option price (by no-arbitrage):
      \begin{align*}
        f_0 &= \Delta S_0 + \Psi\\
        &= e^{-rT} \left\{ \tfrac{e^{rT} - d}{u - d}f_u + \tfrac{u - e^{rT}}{u - d}f_d \right\}
      \end{align*}
  \end{itemize}
\end{frame}

\begin{frame}{Risk-Neutral Pricing}
  \begin{itemize}[<+->]
    \item Risk-neutral probabilities (different from real-world probabilities):
      \begin{align*}
        \pi_u = \tfrac{e^{rT} - d}{u - d}, \quad \pi_d = \tfrac{u - e^{rT}}{u - d}
      \end{align*}
      
    \item For no-arbitrage, must have $d < e^{rT} < u$ (risk-neutral probabilities are positive)
    
    \item Option price as discounted expected value under risk-neutral measure $\mathbb{Q}$:
      \begin{align*}
        f_0 = e^{-rT} \cdot \expc_{\mathbb{Q}}{f_T} = e^{-rT} \cdot (\pi_u f_u + \pi_d f_d)
      \end{align*}
      
    \item Risk-neutral valuation property:
      \begin{align*}
        \expc_{\mathbb{Q}}{S_T} = S_0e^{rT}
      \end{align*}
      The expected return of any asset under $\mathbb{Q}$ is the risk-free rate
      
    \item Martingale property (discounted price process):
      \begin{align*}
        \tfrac{S_0}{B_0} = \expc_{\mathbb{Q}}\left\{\tfrac{S_T}{B_T}\right\}
      \end{align*}
      
    \item Real-world vs. risk-neutral probabilities:
      \begin{itemize}
        \item Real-world: Used for forecasting
        \item Risk-neutral: Used for pricing (incorporates risk premia)
      \end{itemize}
  \end{itemize}
\end{frame}

\begin{frame}{No-Arbitrage Limitations}
  \begin{itemize}[<+->]
    \item Real-life market frictions and imperfections:
      \begin{itemize}
        \item Transaction costs, taxes, and bid-ask spreads create "no-arbitrage bands"
        \item Liquidity issues and margin calls affect dynamic strategies
        \item Short-selling restrictions limit arbitrage implementation
        \item Inventory costs for physical commodities
        \item Securities lending fees and dividends/coupons
      \end{itemize}
      
    \item Market participant limitations:
      \begin{itemize}
        \item Bounded rationality and behavioral biases
        \item Asymmetric information distribution
        \item Noise trader risk (irrational price movements)
        \item Different views on fundamental value
        \item Capital constraints limiting arbitrage capacity
      \end{itemize}
      
    \item Implementation challenges:
      \begin{itemize}
        \item Execution uncertainty (price impact, timing, slippage)
        \item Model risk (wrong specification, parameter estimation errors)
        \item "Limits to arbitrage" and persistence of mispricings
        \item Risk of convergence timing (LTCM example - technically correct but timing wrong)
        \item Correlated positions and crowded trades
      \end{itemize}
  \end{itemize}
\end{frame}

\begin{frame}{The Mathematics of Arbitrage}
  \begin{itemize}[<+->]
    \item Dominant trading strategy exists if $V(0) = 0$, $V(T,\omega) > 0$, $\forall\,\omega\in\Omega$, or equivalently (with risk-free asset) $V(0) < 0$, $V(T,\omega)\geqslant 0$, $\forall\,\omega\in\Omega$
      
    \item Arbitrage opportunity exists if (weaker condition) $V(0) = 0$, $V(T,\omega)\geqslant 0$, $\forall\,\omega\in\Omega$, $\expc{V(T,\omega)} > 0$
      
    \item No arbitrage $\ifff$ Existence of strictly positive pricing kernel/state price density
    
    \item In terms of discounted gains:
      \begin{align*}
        \expc_{\mathbb{Q}}\{G^*(\omega)\} = \sum_{j=1}^m G^*(\omega_j)q(\omega_j) = 0
      \end{align*}
      
    \item Fundamental Theorem of Asset Pricing:
      \begin{itemize}
        \item No arbitrage $\ifff$ Existence of equivalent martingale measure
        \item Market completeness $\ifff$ Uniqueness of equivalent martingale measure
        \item Incomplete markets have multiple martingale measures \\$\ie$ Range of arbitrage-free prices
      \end{itemize}
      
    \item Price of any attainable contingent claim:
      \begin{align*}
        V(0) = \expc_{\mathbb{Q}}\left\{\tfrac{V(T,\omega)}{B(T,\omega)}\right\}
      \end{align*}
  \end{itemize}
\end{frame}

%\begin{frame}[allowframebreaks]
%  \frametitle{References}
%  \nocite{*}
%  \bibliographystyle{apalike}
%  \bibliography{note06}
%\end{frame}

\end{document}
